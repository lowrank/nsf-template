% To edit in Overleaf. Change the first line "\documentclass[./preamble.tex]{subfiles}" to "\documentclass[../preamble.tex]{subfiles}". 

% To create the final version. Just change back.

\documentclass[../preamble.tex]{subfiles}
\graphicspath{{\subfix{./Figures/}}}
\begin{document}

%%%%%%%%%%%%%%%%%%%%%%%%%%%%%%%%%%%%%%%%%%%%%%%%%%%%%%%%%%%%%%%%%%%%%%%%%%%%%%%
%
% The Project Description should outline the general plan of work, including 
% the broad design of activities to be undertaken, and, where appropriate, 
% provide a clear description of experimental methods and procedures. 
% Proposers should address what they want to do, why they want to do it, how 
% they plan to do it, how they will know if they succeed, and what benefits 
% could accrue if the project is successful.
% 
%
% Page Limit: 15 pages. URL cannot be used.
%
% 
%%%%%%%%%%%%%%%%%%%%%%%%%%%%%%%%%%%%%%%%%%%%%%%%%%%%%%%%%%%%%%%%%%%%%%%%%%%%%%%

%%%%%%%%%%%%%%%%%%%%%%%%%%%%%%%%%%%%%%%%%%%%%%%%%%%%%%%%%%%%%%%%%%%%%%%%%%%%%%%
%
%
\subsection{1. INTRODUCTION} % Introduce the background and necessary information. 
%
% 
%
%%%%%%%%%%%%%%%%%%%%%%%%%%%%%%%%%%%%%%%%%%%%%%%%%%%%%%%%%%%%%%%%%%%%%%%%%%%%%%%

In this project, ...

%%%%%%%%%%%%%%%%%%%%%%%%%%%%%%%%%%%%%%%%%%%%%%%%%%%%%%%%%%%%%%%%%%%%%%%%%%%%%%%
\subsection{2. OBJECTIVES OF THE PROJECT}
%
% 
%%%%%%%%%%%%%%%%%%%%%%%%%%%%%%%%%%%%%%%%%%%%%%%%%%%%%%%%%%%%%%%%%%%%%%%%%%%%%%%

In this project, ...

%%%%%%%%%%%%%%%%%%%%%%%%%%%%%%%%%%%%%%%%%%%%%%%%%%%%%%%%%%%%%%%%%%%%%%%%%%%%%%%

%
%
\subsection{3. RESEARCH PLAN} 
%
% Include 1. Objectives
%          2. Significances (why NSF has to fund the project?)
%          3. Relation to the present state (what is new?)
%          4. Outline of the plan (should be brief but informative, provide clear
%                                  description of the method, and procedures)
%          5. How to know if the project succeeds.
%          6. What are the benefits if it succeed?
%%%%%%%%%%%%%%%%%%%%%%%%%%%%%%%%%%%%%%%%%%%%%%%%%%%%%%%%%%%%%%%%%%%%%%%%%%%%%%%

\subsection{Research Objective 1}
In this sub-project, the PI ... see~\cite{knuth2014art} and references therein.

\subsubsection{Subproject 1}
In this subproject.
\subsubsection{Subproject 2}

\subsection{Research Objective 2}

%%%%%%%%%%%%%%%%%%%%%%%%%%%%%%%%%%%%%%%%%%%%%%%%%%%%%%%%%%%%%%%%%%%%%%%%%%%%%%%
%
%
% Education Activities – The education component of the proposal may be in a broad range of areas and may be directed to any level: K-12 students,
% undergraduates, graduate students, and/or the general public, but should be related to the proposed research and consistent with the career goals of the PI.
% Some examples are: incorporating research activities into undergraduate courses; teaching a graduate seminar on the topic of the research; designing
% innovative courses or curricula; providing mentored international research experiences for U.S. students; linking education activities to industrial, international, or
% cross-disciplinary work; supporting teacher preparation and enhancement; conducting outreach and mentoring activities to enhance scientific literacy or involve
% students from groups that have been traditionally underrepresented in science; researching students' learning and conceptual development in the discipline;
% implementing innovative methods for evaluation and assessment; or creating cyberinfrastructure that facilitates involvement of the broad citizenry in the scientific
% enterprise. Education activities may also include designing new or adapting and implementing effective educational materials and practices. Such activities
% should be consistent with research and best practices in curriculum, pedagogy, and evaluation. Proposers may build on, or otherwise meaningfully participate in,
% existing NSF-supported activities or other educational projects ongoing on campus.

\subsection{4. EDUCATION PLAN} % Required.
 The education program detailed below includes ... 
\subsection{Curriculum Development}
\noindent \textit{Undergraduate Course.}

\noindent \textit{Graduate Course.}

\subsection{Postdoc/Graduate/Undergraduate Mentoring}
\noindent \textit{Postdoc.}

\noindent \textit{Graduate.}

\noindent \textit{Undergraduate.}

\subsection{Research Group}

\subsection{Summer School and Workshop}

\subsection{Educational Channel}

%
%
%%%%%%%%%%%%%%%%%%%%%%%%%%%%%%%%%%%%%%%%%%%%%%%%%%%%%%%%%%%%%%%%%%%%%%%%%%%%%%%

%%%%%%%%%%%%%%%%%%%%%%%%%%%%%%%%%%%%%%%%%%%%%%%%%%%%%%%%%%%%%%%%%%%%%%%%%%%%%%%
%
%
\subsection{5. BROADER IMPACTS} % Required.
%
%
%%%%%%%%%%%%%%%%%%%%%%%%%%%%%%%%%%%%%%%%%%%%%%%%%%%%%%%%%%%%%%%%%%%%%%%%%%%%%%%

% a description of other broader impacts, besides the education activities, that will accrue from the project. Impacts on other field, interdisciplinary research, International/Global Dimensions 
The overall objective of this proposal aims at developing new directions of scientific computing for applications with RTE models by bringing the advantages of existing numerical methods together. The potential applications are widespread in many fields such as biomedical imaging, biofuel engineering, national security, etc. The resulting numerical algorithms in this proposal will directly impact these applications and provide fast, efficient, and reliable computational tools. Moreover, the new developments will be helpful to applied mathematicians, biomedical engineers, and practitioners from other disciplines who need to study linear and nonlinear transport models and related computational inverse problems. The PI will adjust the results into undergraduate/graduate scientific computation courses for mathematics, engineering, and biology. This proposal also seeks further collaborations from material science and biofuel engineering and more ideas will evolve as the project proceeds. Besides the educational impacts, the other broader impacts envisioned are outlined below.
\subsection{Interdisciplinary Perspectives}
\subsection{Career Preparations for Postdoc/Graduate/Undergraduate}
\subsection{Software Development and Data Dissemination}

%%%%%%%%%%%%%%%%%%%%%%%%%%%%%%%%%%%%%%%%%%%%%%%%%%%%%%%%%%%%%%%%%%%%%%%%%%%%%%%
%
%
\subsection{6. RESULTS FROM PRIOR NSF SUPPORT} % Required.
%
% Includes:
%     - an award with an end date in the past *five* years; or
%     - any current funding, including any no-cost extensions.
% 
% Information on the award is required for each PI and co-PI. 
% Refer to PAPPG d. (iii)
%
%
%%%%%%%%%%%%%%%%%%%%%%%%%%%%%%%%%%%%%%%%%%%%%%%%%%%%%%%%%%%%%%%%%%%%%%%%%%%%%%%
\begin{enumerate}
    \item Award number, amount, period of support, title. 
    \begin{itemize}
        \item      Project summary
        \item      Publications
        \item      Evidence of research availability.
        \item      If for renewed support, description of the relation of the old work and new work.
    \end{itemize}
     
\end{enumerate}

\newpage

%%%%%%%%%%%% Refereces Cited %%%%%%%%%%%%%%%%%%%%
%
%


\AtBeginShipout{%
\AtBeginShipoutDiscard
}
\renewcommand\refname{References Cited}
\bibliography{references}
% Any referencing style is applicable.
\bibliographystyle{ieeetr}




%%%%%%%%%%%%%%%%%%%%%%%%%%%%%%%%%%%%%%%%%%%%%%%%%%%%%%%%%%%%%%%%%%%%%%%%%%%%%%%
%%%%%%%%%%%%%%%%%%%%%%%%%%%%%%%%%%%%%%%%%%%%%%%%%%%%%%%%%%%%%%%%%%%%%%%%%%%%%%%
\end{document}
