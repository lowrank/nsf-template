\documentclass[../preamble.tex]{subfiles}
\graphicspath{{\subfix{./Figures/}}}
\begin{document}
\section{MENTORING PLAN}
%%%%%%%%%%%%%%%%%%%%%%%%%%%%%%%%%%%%%%%%%%%%%%%%%%%%%%%%%%%%%%%%%%%%%%%%%%%%%%%
% one page mentoring plan for postdoc and graduate students


% The goal of the mentoring plan is to provide the skills, knowledge, and experiences
% necessary to prepare postdoctoral researchers and graduate students to excel in their 
% chosen career path. Examples of mentoring activities include but are not limited to: 
% career counseling; training in preparation of proposals, publications and presentations; 
% guidance on ways to improve teaching and mentoring skills; guidance on how to effectively
% collaborate with researchers from diverse backgrounds and disciplinary areas; 
% and training in responsible professional practices.


%%%%%%%%%%%%%%%%%%%%%%%%%%%%%%%%%%%%%%%%%%%%%%%%%%%%%%%%%%%%%%%%%%%%%%%%%%%%%%%
This Mentoring Plan establishes guidelines for a Postdoctoral Researcher or Graduate Student in support of the NSF Project Awarded to <institution name>, entitled ``<title of project>''. 
\vpn\rule{\textwidth}{0.4pt}
\subsection{Orientation} %Orientation topics will include (a) the amount of independence the Postdoctoral Researcher requires, (b) interaction with coworkers, (c) productivity including the importance of scientific publications, (d) work habits and laboratory safety, and (e) documentation of research methodologies and experimental details so that the work can be continued by other researchers in the future.
\subsection{Career Counseling/Advising} %will be directed at providing the Postdoctoral Researcher with the skills, knowledge, and experience needed to excel in his/her chosen career path. In addition to guidance provided by <post doc researcher name>, the Postdoctoral Researcher will be encouraged to discuss career options with researchers and managers at <university name> and with former students and colleagues of < post doc researcher name>.
\subsection{Training in Preparation of Grant Proposals} %will be gained by direct involvement of the Postdoctoral Researcher in proposals prepared by <company name>. The Postdoctoral Researcher will have an opportunity to learn best practices in proposal preparation including identification of key research questions, definition of objectives, description of approach and rationale, and construction of a work plan, timeline, and budget.
\subsection{Publications and Presentations} %Tare expected to result form the work supported by the grant. These will be prepared under the direction of < post doc researcher name> and in collaboration with researchers at <instituion name> as appropriate. The Postdoctoral Researcher will receive guidance and training in the preparation of manuscripts for scientific journals and presentations at conferences. 
\subsection{Teaching and Mentoring Skills} %will be developed in the context of regular meetings within <university name> research group during which graduate students and postdoctoral researchers describe their work to colleagues within the group and assist each other with solutions to challenging research problems, often resulting in cross fertilization of ideas.
\subsection{Instruction in Responsible Professional Practices} %will be provided on a regular basis in the context of the research work and will include fundamentals of the scientific method, laboratory safety, and other standards of professional practice. In addition, the Postdoctoral Researcher will be encouraged to affiliate with one or more professional societies in his/her chosen field.
\subsection{Success of Mentoring Plan} %will be assessed by monitoring the personal progress of the Postdoctoral Researcher through a tracking of the Postdoctoral Researcher’s progress toward his/her career goals after finishing the postdoctoral program.

\end{document}